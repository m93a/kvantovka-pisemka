% !TEX program = xelatex

\documentclass[10pt,a4paper]{article}
\usepackage[top = 1.5cm, bottom = 1.5cm, left = 1.5cm, right = 1.5cm]{geometry}

\usepackage{titling}
\usepackage[czech]{babel}
\usepackage{graphicx}
\usepackage{lmodern}
\usepackage{hyperref}
\usepackage{setspace}
\usepackage{csvsimple}

\usepackage{amsmath}
\usepackage{amssymb}
\usepackage{gensymb}
\usepackage{units}
\usepackage{bm}
\delimitershortfall=-1pt

\usepackage{gnuplottex}
\usepackage{epstopdf}

% no page break
\newenvironment{absolutelynopagebreak}
  {\par\nobreak\vfil\penalty0\vfilneg
   \vtop\bgroup}
  {\par\xdef\tpd{\the\prevdepth}\egroup
   \prevdepth=\tpd}


% redefine \sqrt
\usepackage{letltxmacro}
\makeatletter
\let\oldr@@t\r@@t
\def\r@@t#1#2{%
\setbox0=\hbox{$\oldr@@t#1{#2\,}$}\dimen0=\ht0
\advance\dimen0-0.2\ht0
\setbox2=\hbox{\vrule height\ht0 depth -\dimen0}%
{\box0\lower0.4pt\box2}}
\LetLtxMacro{\oldsqrt}{\sqrt}
\renewcommand*{\sqrt}[2][\ ]{\oldsqrt[#1]{#2\,}\,}
\makeatother

\def\ph{\phantom}
\def\vph{\vphantom}
\def\hph{\hphantom}

\def\?{\mathit{?}}

\newcommand{\comm}[2]{\left[ #1, #2 \right]}
\newcommand{\const}[1]{\text{#1}}
\newcommand{\norm}[1]{\left\lVert#1\right\rVert}

\newcommand{\mat}[1]{
    \begin{pmatrix}
        #1
    \end{pmatrix}
}

\newcommand{\mata}[2]{
    \left(
    \begin{array}{@{}#1@{}}
        #2
    \end{array}
    \right)
}

\newcommand{\smat}[2][1]{
    \scalebox{#1}{$\mat{#2}$}
}

\renewcommand{\d}[1]{\;\const{d}#1}
\newcommand{\dd}[2]{\frac{\const{d} #1}{\const{d} #2} \;}
\newcommand{\pd}[2]{\frac{\partial  #1}{\partial  #2} \;}

\newcommand{\bra}[1]{\left< #1 \right|}
\newcommand{\ket}[1]{\left| #1 \right>}
\newcommand{\braket}[2]{\left< #1 \middle| #2 \right>}

\newcommand{\e}[1]{\const{e}^{#1}}
\renewcommand{\i}{\const{i}}

\newcommand*\Laplace{\mathop{}\!\mathbin\bigtriangleup}

\begin{document}

\title{Kvantová mechanika I: Zápočtová písemka}
\author{Michal Grňo}
\date{\today}

\maketitle

\section{Mionium v křemíku}



\section{Kvadratický Hamiltonián}
\subsection{Zadání}
Pohyb částice je popsán Hamiltoniánem
\begin{align*}
    \hat{H}
    = \hat{p}^4
    + \hat{p}^2 \hat{x}^2
    + \hat{x}^2 \hat{p}^2
    + \hat{x}^4,
    \hspace{2em}
    \comm{\hat{x}}{\hat{p}} = \i.
\end{align*}
Určete možné energie $E_n$ této částice a jim odpovídající vlastní stavy $\ket{n}$. Vypočtěte střední hodnotu operátoru souřadnice v čase $t$, pokud je systém na počátku v čase $t=0$ ve stavu
\begin{align*}
    \ket{\psi} =
    \frac{1}{\oldsqrt{2}}
    \left( \ket{n} + \ket{n+1} \right).
\end{align*}

\subsection{Řešení}
Ze cvičení víme, že Hamiltonián harmonického oscilátoru je pro $M=\frac{1}{2}, \; \Omega = 1, \; \hbar = 1$ ve tvaru
\begin{align*}
    \hat{H}_\textit{harm.} = \hat{p}^2 + \hat{x}^2
\end{align*}
si můžeme vyjádřit jako
\begin{gather*}
    \hat{H}_\textit{harm.} = \hat{n} + \frac{1}{2},
    \\[5pt]
    \hat{n} = \hat{a}^\dagger \hat{a},
    \hspace{1em}
    \hat{n} \ket{n} =  n \ket{n}
    \;\; \forall n \in \mathbb{N}_0,
    \\[5pt]
    \hat{x} = \hat{a}^\dagger + \hat{a},
    \hspace{2em}
    \hat{p} = \i \left( \hat{a}^\dagger - \hat{a} \right),
    \\[5pt]
    \hat{a} \ket{n} = \sqrt{n} \ket{n-1},
    \\[5pt]
    \hat{a}^\dagger \ket{n} = \sqrt{n+1} \ket{n+1}.
\end{gather*}
Pro náš kvadratický Hamiltonián zjevně platí
\begin{align*}
    \hat{H}
    &= \hat{p}^4
    + \hat{p}^2 \hat{x}^2
    + \hat{x}^2 \hat{p}^2
    + \hat{x}^4,
    \\[5pt]
    &= \left( \hat{p}^2 + \hat{x}^2 \right)^2 \\[5pt]
    &= \left( \hat{n} + \frac{1}{2} \right)^2 \\[5pt]
    &= \hat{n}^2 + \hat{n} + \frac{1}{2}.
\end{align*}
Jeho vlastní stavy budou tedy $\ket{n}$, stejné jako vlastní stavy harmonického oscilátoru, a jim příslušící vlastní energie budou
\begin{gather*}
    E_n \ket{n}
    = \hat{H} \ket{n}
    = \left( \hat{n}^2 + \hat{n} + \frac{1}{2} \right) \ket{n}
    = \left( n^2 + n + \frac{1}{2} \right) \ket{n}.
    \\[15pt]
    E_n = n^2 + n + \frac{1}{2}.
\end{gather*}


Nakonec nás zajímá časový vývoj stavu $\ket{\psi(t)}$ s počáteční podmínkou
\begin{align*}
    \ket{\psi(0)} =
    \frac{1}{\oldsqrt{2}}
    \left( \ket{n} + \ket{n+1} \right).
\end{align*}
Víme, že pro časový vývoj obecného stavu platí (při $\hbar = 1$)
\begin{align*}
    \ket{\psi(t)} = \e{-\i \hat{H} t} \ket{\psi(0)}.
\end{align*}
S použitím Sylvestrovy formule pro rozklad operátorové funkce pomocí vlastních stavů dostaneme
\begin{align*}
    \ket{\psi(t)} &= \sum_m \e{-\i E_m t} \hat{\const{P}}_{E_m} \ket{\psi(0)} = \frac{1}{\oldsqrt{2}} \left(\e{-\i E_n t} \ket{n} + \e{-\i E_{n+1} t} \ket{n+1}\right),
    \\[10pt]
    \ket{\psi(t)} &= \frac{1}{\oldsqrt{2}} \e{-\i t \left(n^{2} + n + \frac{1}{2}\right)} \left(\ket{n} + \e{-2 \i t \left(n + 1\right)} \ket{n+1}\right).
\end{align*}
Střední hodnota operátoru souřadnice $\hat{x}$ potom bude
\begin{align*}
    \bra{\psi(t)} \; \hat{x} \; \ket{\psi(t)}
    &= \left(\bra{n} + \e{2 \i t \left(n + 1\right)} \bra{n+1}\right) \; \hat{x} \; \left(\ket{n} + \e{-2 \i t \left(n + 1\right)} \ket{n+1}\right)
    \\[5pt]
    &= \e{2 \i t \left(n + 1\right)} {\left\langle n + 1\right|} \hat{x} \ket{n} + {\left\langle n\right|} \hat{x} \ket{n} + {\left\langle n + 1\right|} \hat{x} \ket{n+1} + \e{- 2 \i t \left(n + 1\right)} {\left\langle n\right|} \hat{x} \ket{n+1}
    \\[5pt]
    &= \e{2 \i t \left(n + 1\right)} {\left\langle n + 1\right|} \left(\hat{a}^{\dagger} + \hat{a}\right) \ket{n} + {\left\langle n\right|} \left(\hat{a}^{\dagger} + \hat{a}\right) \ket{n} + {\left\langle n + 1\right|} \left(\hat{a}^{\dagger} + \hat{a}\right) \ket{n+1} + \e{- 2 \i t \left(n + 1\right)} {\left\langle n\right|} \left(\hat{a}^{\dagger} + \hat{a}\right) \ket{n+1}
    \\[5pt]
    &= \sqrt{n + 1} \e{2 \i t} \e{2 \i n t} + \sqrt{n + 1} \e{- 2 \i t} \e{- 2 \i n t}
    \\[5pt]
    &= 2 \sqrt{n + 1} \cos{\left(2 t \left(n + 1\right) \right)}.
\end{align*}
Použil jsem vyjádření $\hat{x}$ pomocí posunovacích operátorů a následné zjednodušení pomocí knihovny SymPy.




\section{Třírozměrná \texorpdfstring{$\delta$}{δ} jáma}

\section{Zadání}
Částice se pohybuje v třírozměrném přitažlivém potenciálu
\begin{align*}
    V(x,y,z) = - a \delta(x) - b \delta(y) - c \delta(z),
    \hspace{2em}
    a,b,c \; > \; 0.
\end{align*}
Nalezněte energie všech vázaných stavů a jim odpovídající normované vlnové funkce. Pro nejnižší vázaný stav ověřte, že splňuje relace neurčitosti.

\section{Řešení}
Odpovídající Hamiltonián je pro $M = \frac{1}{2}, \; \Omega = 1, \; \hbar = 1$
\begin{align*}
    \hat{H} = \hat{p}^2 + \hat{V} = -\Laplace - a \delta(x) - b \delta(y) - c \delta(z)
\end{align*}




\end{document}
