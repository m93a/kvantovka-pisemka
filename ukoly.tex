% !TEX program = xelatex

\documentclass[10pt,a4paper]{article}
\usepackage[top = 1.5cm, bottom = 1.5cm, left = 1.5cm, right = 1.5cm]{geometry}

\usepackage{titling}
\usepackage[czech]{babel}
\usepackage{graphicx}
\usepackage{lmodern}
\usepackage{hyperref}
\usepackage{setspace}
\usepackage{csvsimple}

\usepackage{amsmath}
\usepackage{amssymb}
\usepackage{gensymb}
\usepackage{units}
\usepackage{bm}
\delimitershortfall=-1pt

\usepackage{gnuplottex}
\usepackage{epstopdf}

% no page break
\newenvironment{absolutelynopagebreak}
  {\par\nobreak\vfil\penalty0\vfilneg
   \vtop\bgroup}
  {\par\xdef\tpd{\the\prevdepth}\egroup
   \prevdepth=\tpd}


% redefine \sqrt
\usepackage{letltxmacro}
\makeatletter
\let\oldr@@t\r@@t
\def\r@@t#1#2{%
\setbox0=\hbox{$\oldr@@t#1{#2\,}$}\dimen0=\ht0
\advance\dimen0-0.2\ht0
\setbox2=\hbox{\vrule height\ht0 depth -\dimen0}%
{\box0\lower0.4pt\box2}}
\LetLtxMacro{\oldsqrt}{\sqrt}
\renewcommand*{\sqrt}[2][\ ]{\oldsqrt[#1]{#2\,}\,}
\makeatother

\def\ph{\phantom}
\def\vph{\vphantom}
\def\hph{\hphantom}

\def\?{\mathit{?}}

\newcommand{\comm}[2]{\left[ #1, #2 \right]}
\newcommand{\const}[1]{\text{#1}}
\newcommand{\norm}[1]{\left\lVert#1\right\rVert}

\newcommand{\mat}[1]{
    \begin{pmatrix}
        #1
    \end{pmatrix}
}

\newcommand{\mata}[2]{
    \left(
    \begin{array}{@{}#1@{}}
        #2
    \end{array}
    \right)
}

\newcommand{\smat}[2][1]{
    \scalebox{#1}{$\mat{#2}$}
}

\renewcommand{\d}[1]{\;\const{d}#1}
\newcommand{\dd}[2]{\frac{\const{d} #1}{\const{d} #2} \;}
\newcommand{\pd}[2]{\frac{\partial  #1}{\partial  #2} \;}

\newcommand{\bra}[1]{\left< #1 \right|}
\newcommand{\ket}[1]{\left| #1 \right>}
\newcommand{\braket}[2]{\left< #1 \middle| #2 \right>}

\newcommand{\e}[1]{\const{e}^{#1}}
\renewcommand{\i}{\const{i}}

\newcommand*\Laplace{\mathop{}\!\mathbin\bigtriangleup}

\begin{document}

\title{Kvantová mechanika I: Zápočtová písemka}
\author{Michal Grňo}
\date{\today}

\maketitle

\section{Mionium v křemíku}

\section{Zadání}
Mionium je vázaný stav mionu s elektronem, podobá se např. atomu vodíku, lze ho vytvořit např. bombardováním křemíkového krystalu miony. Interakce spinu mionu se spinem elektronu a interakce mionia v základním stavu s krystalem se dá modelovat Hamiltonián
\begin{gather*}
    \hat{H} + \hat{E}_0 + \frac{A}{\hbar^2} \hat{\bm{s}}_\mu \cdot \hat{\bm{s}}_e + \frac{D}{\hbar^2} \hat{s}_{\mu 3} \hat{s}_{e 3},
    \\[10pt]
    \hat{E}_0 = - \tfrac{1}{2} m_r c^2 \alpha^2 \;\; \hat{1},
    \\[10pt]
    \hat{\bm{s}}_\mu = \tfrac{\hbar}{2} \hat{\bm{\sigma}} \otimes \hat{1},
    \hspace{1em}
    \hat{\bm{s}}_e = \hat{1} \otimes \tfrac{\hbar}{2} \hat{\bm{\sigma}},
\end{gather*}
kde $\bm{\sigma} = (\sigma_1, \sigma_2, \sigma_3)$ jsou Pauliho matice.

\section{Řešení}
Pauliho matice jsou
\begin{align*}
    \sigma_1 = \smat[0.7]{0 & 1\\1 & 0}
    \hspace{2em}
    \sigma_2 = \smat[0.7]{0 & - \i\\\i & 0}
    \hspace{2em}
    \sigma_3 = \smat[0.7]{1 & 0\\0 & -1}
\end{align*}
Přímým dosazením do rovnice Hamiltoniánu tedy dostaneme:
\begin{align*}
    E_0 =
    - \frac{\alpha^{2} c^{2} m_{r}}{2}
    \smat[0.7]{1 \\ & 1 \\ & & 1 \\ & & & 1}
    \hspace{3em}
    \bm{s}_\mu \cdot \bm{s}_e =
    \frac{\hbar^2}{4}
    \smat[0.7]{1 & & & \\ & -1 & 2 & \\ & 2 & -1 & \\ & & & 1}
    \hspace{3em}
    s_{\mu 3} s_{e 3} =
    \frac{\hbar^2}{4}
    \smat[0.7]{1 & & & \\ & -1 &  & \\ &  & -1 & \\ &  &  & 1}
\end{align*}
\begin{align*}
    H = \frac{1}{4}
    \mat{A + D - 2 \alpha^{2} c^{2} m_{r} & 0 & 0 & 0\\0 & - A - D - 2 \alpha^{2} c^{2} m_{r} & 2 A & 0\\0 & 2 A & - A - D - 2 \alpha^{2} c^{2} m_{r} & 0\\0 & 0 & 0 & A + D - 2 \alpha^{2} c^{2} m_{r}}
\end{align*}
Diagonalizujeme-li $H$, získáme
\begin{gather*}
    H = R^{-1} H_\textit{diag.} R
    \\[10pt]
    R = \mat{0 & - \nicefrac{\oldsqrt{2}}{2} & \nicefrac{\oldsqrt{2}}{2} & 0\\0 & \nicefrac{\oldsqrt{2}}{2} & \nicefrac{\oldsqrt{2}}{2} & 0\\1 & 0 & 0 & 0\\0 & 0 & 0 & 1}
    \\[10pt]
    H_\textit{diag.} = \mat{- \frac{3 A}{4} - \frac{D}{4} - \frac{\alpha^{2} c^{2} m_{r}}{2} & 0 & 0 & 0\\0 & \frac{A}{4} - \frac{D}{4} - \frac{\alpha^{2} c^{2} m_{r}}{2} & 0 & 0\\0 & 0 & \frac{A}{4} + \frac{D}{4} - \frac{\alpha^{2} c^{2} m_{r}}{2} & 0\\0 & 0 & 0 & \frac{A}{4} + \frac{D}{4} - \frac{\alpha^{2} c^{2} m_{r}}{2}}
\end{gather*}
Na diagonále $H_\textit{diag}$ jsou energie vázaných stavů Hamiltoniánu, sloupce $R$ jsou jim odpovídající stavové vektory~${\ket{\psi_\mu} \otimes \ket{\psi_e}}$.

Jako podúkol máme vyjádřit $\ket{\rightarrow}^{(\mu)}$ v bázi $\{ \ket{\uparrow}^{(\mu)}, \ket{\downarrow}^{(\mu)} \}$. Ze cvičení víme, že těmito \textit{kety} se označují vlastní vektory Pauliho matic:
\begin{align*}
    \ket{\rightarrow} = \frac{1}{\oldsqrt{2}} \mat{1 \\ 1},
    \hspace{2em}
    \ket{\uparrow} = \mat{1 \\ 0},
    \hspace{2em}
    \ket{\downarrow} = \mat{0 \\ 1}.
\end{align*}
Vyjádření $\ket{\rightarrow}^{(\mu)}$ v bázi $\{ \ket{\uparrow}^{(\mu)}, \ket{\downarrow}^{(\mu)} \}$ je tedy zřejmě:
\begin{align*}
    \ket{\rightarrow}^{(\mu)} = \frac{1}{\oldsqrt{2}} \left(\ket{\uparrow}^{(\mu)} + \ket{\downarrow}^{(\mu)} \right).
\end{align*}

Dále předpokládáme, že v čase $t=0$ je spin elektronu $\ket{\psi}^{(e)} = \ket{\uparrow}^{(e)}$, zatímco spin mionu je v obecném stavu:
\begin{gather*}
    \ket{\psi(0)}^{(e)} = \mat{1 \\ 0},
    \hspace{2em}
    \ket{\psi(0)}^{(\mu)} = \mat{a \\ b},
    \\[10pt]
    \ket{\psi(0)} = \ket{\psi(0)}^{(e)} \otimes \ket{\psi(0)}^{(\mu)}
    = \mat{a \\ 0 \\ b \\ 0}.
\end{gather*}
Vyjádříme-li si tento vektor pomocí vlastních stavů Hamiltoniánu $\ket{n} = R_n$, dostaneme:
\begin{align*}
    \ket{\psi(0)} = \frac{a}{\oldsqrt{2}} \left( \; -\ket{2} + \ket{3} \; \right) + b \ket{1}.
\end{align*}
Víme, že pro časový vývoj obecného stavu platí:
\begin{align*}
    \ket{\psi(t)} = \e{-\i \hat{H} t/\hbar} \ket{\psi(0)}.
\end{align*}
S použitím Sylvestrovy formule pro rozklad operátorové funkce pomocí vlastních stavů dostaneme
\begin{align*}
    \ket{\psi(t)} &= \sum_m \e{-\i E_m t/\hbar} \hat{\const{P}}_{E_m} \ket{\psi(0)}
    \\
    &= b \e{-\i E_1 t/\hbar} \; \ket{1} \;\;\; - \frac{a}{\oldsqrt{2}} \e{-\i E_2 t/\hbar} \; \ket{2} \;\;\; + \frac{a}{\oldsqrt{2}} \e{-\i E_3 t} \; \ket{3},
\end{align*}
\begin{gather*}
    \ket{\psi(t)} =
    \e{-\i (A + D - 2  \alpha^2 c^2  m_r) t / 4\hbar}
    \left(
        b \e{ \i (2A + D) t / 2\hbar} \ket{1} +
        \frac{a}{\oldsqrt{2}}
        \left( -\e{\i D t / 2\hbar} \ket{2} + \ket{3} \right)
    \right).
\end{gather*}
Nyní dosadíme zpět za vlastní vektory $\ket{n}$ a získáme:
\begin{gather*}
    \ket{\psi(t)}
    =
    \mat{\frac{a \left(e^{\frac{i D t}{2 \hbar}} + 1\right) e^{- \frac{i t \left(A + D - 2 \alpha^{2} c^{2} m_{r}\right)}{4 \hbar}}}{2}\\\frac{a \left(1 - e^{\frac{i D t}{2 \hbar}}\right) e^{- \frac{i t \left(A + D - 2 \alpha^{2} c^{2} m_{r}\right)}{4 \hbar}}}{2}\\b e^{\frac{i t \left(3 A + D + 2 \alpha^{2} c^{2} m_{r}\right)}{4 \hbar}}\\0}
    =
    \mat{
        \psi_{\uparrow_\mu\uparrow_e}(t) \\[5pt]
        \psi_{\uparrow_\mu\downarrow_e}(t) \\[5pt]
        \psi_{\downarrow_\mu\uparrow_e}(t) \\[5pt]
        \psi_{\downarrow_\mu\downarrow_e}(t)
    }
\end{gather*}
Všimněme si, že první dvě složky obě odpovídají stavu, kdy je spin mionu ve směru $\uparrow$, platí tedy:
\begin{align*}
    P(\uparrow_\mu) = P(\uparrow_\mu \uparrow_e) + P(\uparrow_\mu \downarrow_e) = \int_{a,b} |\psi_{\uparrow_\mu\uparrow_e}(t)|^2 + |\psi_{\uparrow_\mu\downarrow_e}(t)|^2 \d{P},
\end{align*}
kde $\int_{a,b} \d{P}$ je integrál přes všechny možné počáteční hodnoty $a, b$, $a^2 + b^2 = 1$.



\section{Kvadratický Hamiltonián}
\subsection{Zadání}
Pohyb částice je popsán Hamiltoniánem
\begin{align*}
    \hat{H}
    = \hat{p}^4
    + \hat{p}^2 \hat{x}^2
    + \hat{x}^2 \hat{p}^2
    + \hat{x}^4,
    \hspace{2em}
    \comm{\hat{x}}{\hat{p}} = \i.
\end{align*}
Určete možné energie $E_n$ této částice a jim odpovídající vlastní stavy $\ket{n}$. Vypočtěte střední hodnotu operátoru souřadnice v čase $t$, pokud je systém na počátku v čase $t=0$ ve stavu
\begin{align*}
    \ket{\psi} =
    \frac{1}{\oldsqrt{2}}
    \left( \ket{n} + \ket{n+1} \right).
\end{align*}

\subsection{Řešení}
Ze cvičení víme, že Hamiltonián harmonického oscilátoru je pro $M=\frac{1}{2}, \; \Omega = 1, \; \hbar = 1$ ve tvaru
\begin{align*}
    \hat{H}_\textit{harm.} = \hat{p}^2 + \hat{x}^2
\end{align*}
si můžeme vyjádřit jako
\begin{gather*}
    \hat{H}_\textit{harm.} = \hat{n} + \frac{1}{2},
    \\[5pt]
    \hat{n} = \hat{a}^\dagger \hat{a},
    \hspace{1em}
    \hat{n} \ket{n} =  n \ket{n}
    \;\; \forall n \in \mathbb{N}_0,
    \\[5pt]
    \hat{x} = \hat{a}^\dagger + \hat{a},
    \hspace{2em}
    \hat{p} = \i \left( \hat{a}^\dagger - \hat{a} \right),
    \\[5pt]
    \hat{a} \ket{n} = \sqrt{n} \ket{n-1},
    \\[5pt]
    \hat{a}^\dagger \ket{n} = \sqrt{n+1} \ket{n+1}.
\end{gather*}
Pro náš kvadratický Hamiltonián zjevně platí
\begin{align*}
    \hat{H}
    &= \hat{p}^4
    + \hat{p}^2 \hat{x}^2
    + \hat{x}^2 \hat{p}^2
    + \hat{x}^4,
    \\[5pt]
    &= \left( \hat{p}^2 + \hat{x}^2 \right)^2 \\[5pt]
    &= \left( \hat{n} + \frac{1}{2} \right)^2 \\[5pt]
    &= \hat{n}^2 + \hat{n} + \frac{1}{2}.
\end{align*}
Jeho vlastní stavy budou tedy $\ket{n}$, stejné jako vlastní stavy harmonického oscilátoru, a jim příslušící vlastní energie budou
\begin{gather*}
    E_n \ket{n}
    = \hat{H} \ket{n}
    = \left( \hat{n}^2 + \hat{n} + \frac{1}{2} \right) \ket{n}
    = \left( n^2 + n + \frac{1}{2} \right) \ket{n}.
    \\[15pt]
    E_n = n^2 + n + \frac{1}{2}.
\end{gather*}


Nakonec nás zajímá časový vývoj stavu $\ket{\psi(t)}$ s počáteční podmínkou
\begin{align*}
    \ket{\psi(0)} =
    \frac{1}{\oldsqrt{2}}
    \left( \ket{n} + \ket{n+1} \right).
\end{align*}
Víme, že pro časový vývoj obecného stavu platí (při $\hbar = 1$)
\begin{align*}
    \ket{\psi(t)} = \e{-\i \hat{H} t} \ket{\psi(0)}.
\end{align*}
S použitím Sylvestrovy formule pro rozklad operátorové funkce pomocí vlastních stavů dostaneme
\begin{align*}
    \ket{\psi(t)} &= \sum_m \e{-\i E_m t} \hat{\const{P}}_{E_m} \ket{\psi(0)} = \frac{1}{\oldsqrt{2}} \left(\e{-\i E_n t} \ket{n} + \e{-\i E_{n+1} t} \ket{n+1}\right),
    \\[10pt]
    \ket{\psi(t)} &= \frac{1}{\oldsqrt{2}} \e{-\i t \left(n^{2} + n + \frac{1}{2}\right)} \left(\ket{n} + \e{-2 \i t \left(n + 1\right)} \ket{n+1}\right).
\end{align*}
Střední hodnota operátoru souřadnice $\hat{x}$ potom bude
\begin{align*}
    \bra{\psi(t)} \; \hat{x} \; \ket{\psi(t)}
    &= \left(\bra{n} + \e{2 \i t \left(n + 1\right)} \bra{n+1}\right) \; \hat{x} \; \left(\ket{n} + \e{-2 \i t \left(n + 1\right)} \ket{n+1}\right)
    \\[5pt]
    &= \e{2 \i t \left(n + 1\right)} {\left\langle n + 1\right|} \hat{x} \ket{n} + {\left\langle n\right|} \hat{x} \ket{n} + {\left\langle n + 1\right|} \hat{x} \ket{n+1} + \e{- 2 \i t \left(n + 1\right)} {\left\langle n\right|} \hat{x} \ket{n+1}
    \\[5pt]
    &= \e{2 \i t \left(n + 1\right)} {\left\langle n + 1\right|} \left(\hat{a}^{\dagger} + \hat{a}\right) \ket{n} + {\left\langle n\right|} \left(\hat{a}^{\dagger} + \hat{a}\right) \ket{n} + {\left\langle n + 1\right|} \left(\hat{a}^{\dagger} + \hat{a}\right) \ket{n+1} + \e{- 2 \i t \left(n + 1\right)} {\left\langle n\right|} \left(\hat{a}^{\dagger} + \hat{a}\right) \ket{n+1}
    \\[5pt]
    &= \sqrt{n + 1} \e{2 \i t} \e{2 \i n t} + \sqrt{n + 1} \e{- 2 \i t} \e{- 2 \i n t}
    \\[5pt]
    &= 2 \sqrt{n + 1} \cos{\left(2 t \left(n + 1\right) \right)}.
\end{align*}
Použil jsem vyjádření $\hat{x}$ pomocí posunovacích operátorů a následné zjednodušení pomocí knihovny SymPy.




\section{Třírozměrná \texorpdfstring{$\delta$}{δ} jáma}

\section{Zadání}
Částice se pohybuje v třírozměrném přitažlivém potenciálu
\begin{align*}
    V(x,y,z) = - a \delta(x) - b \delta(y) - c \delta(z),
    \hspace{2em}
    a,b,c \; > \; 0.
\end{align*}
Nalezněte energie všech vázaných stavů a jim odpovídající normované vlnové funkce. Pro nejnižší vázaný stav ověřte, že splňuje relace neurčitosti.

\section{Řešení}
Odpovídající Hamiltonián je pro $M = \frac{1}{2}, \; \Omega = 1, \; \hbar = 1$
\begin{align*}
    \hat{H} = \hat{p}^2 + \hat{V} = -\Laplace - a \delta(x) - b \delta(y) - c \delta(z)
\end{align*}




\end{document}
